
\input{/Users/stille/GitHub/LaTeX-Template/presentation/presentation_1.tex}


%----------------------------------------------------------------------------------------
%	TITLE PAGE
%----------------------------------------------------------------------------------------

\title[Slides for Power Reading Group]{Stochastic Joint Optimization of Wind Generation and Pumped-Storage Units in an Electricity Market} % The short title appears at the bottom of every slide, the full title is only on the title page

\author{Xinbo Geng} % Your name
\institute[TAMU] % Your institution as it will appear on the bottom of every slide, may be shorthand to save space
{
Texas A\&M University\\ % Your institution for the title page
\medskip
\textit{gengxinbo@gmail.com} % Your email address
}
\date{\today} % Date, can be changed to a custom date

\begin{document}

\begin{frame}
\titlepage % Print the title page as the first slide
\end{frame}

\begin{frame}
\frametitle{Overview} % Table of contents slide, comment this block out to remove it
\tableofcontents % Throughout your presentation, if you choose to use \section{} and \subsection{} commands, these will automatically be printed on this slide as an overview of your presentation
\end{frame}


%----------------------------------------------------------------------------------------
%	PRESENTATION SLIDES
%----------------------------------------------------------------------------------------

\section{Background}
\begin{frame}
\frametitle{Renewables, Problems, Solution}
\begin{itemize}
\item Increasing concern over the environmental impact
\item Renewable energy sources become an important portion of generation mix.
\item Renewables participate in electricity market competition.
\item Wind Energy is the fast growing renewable energy.
\end{itemize}
\begin{tcolorbox}[colback=red!5,colframe=red!40!black,title= Drawbacks of Wind]
  \begin{itemize}
  \item uncontrollable
  \item uncertain
  \item fluctuated
  \end{itemize}
\end{tcolorbox}
\begin{tcolorbox}[colback=green!5,colframe=green!40!black,title= Solution]
the Joint Operation of wind power and pumped-storage units in electricity market.
\end{tcolorbox}
\end{frame}
\begin{frame}
\frametitle{Renewables in the Spot Market}
\begin{columns}[c]
\column{0.5\linewidth}
   \includegraphics[width = \linewidth]{mrkt_schedule.png}

\column{0.5\linewidth}
For a wind producer in market:
  \begin{enumerate}
  \item[BF 10] forecast his production, build bids (for Day-ahead Market)
  \item[BF 10] submit hourly bids for the following day
  \item[D+1] (optional) submit balancing bids in real-time market
  \item[D+1] optimize its operation/output
  \item[D+1] Penalty because of the output deviation from the bids
  \end{enumerate}
\end{columns}
\end{frame}
%----------------------------------------------------------------------------------------
\section{Method}
\subsection{Stochastic Programming}
\begin{frame}
\frametitle{Stochastic Programming}
\begin{tcolorbox}[colback=red!5,colframe=red!40!black,title= Uncertainty]
Day-ahead forecast error of wind power is huge (30\% - 50\%)

Also uncertainty about market price.
\end{tcolorbox}
\begin{tcolorbox}[colback=green!5,colframe=green!40!black,title= Solution - Stochastic Programming]
Stochastic Programming deals with the problems of making optimal decisions under uncertainty.
\end{tcolorbox}
\end{frame}

\begin{frame}
\frametitle{Two Stage Stochastic Programming}
\begin{columns}
\column{0.5\linewidth}
  \includegraphics[width = 1.1\linewidth]{2_stage.png}
\column{0.5\linewidth}
\begin{description}
\item[1st-Stage] Hourly bids submitted to the day-ahead market
\item[2nd-Stage] Operation of the wind power and pumped-storage units
\end{description}
\end{columns}
\end{frame}



%----------------------------------------------------------------------------------------
\section{Model}
\subsection{Uncoordinated Operation (UO)}
\begin{frame}
  \frametitle{Uncoordinated Operation (UO)}
\begin{tcolorbox}[colback=white!5,colframe=white!40!black,title= Model of Wind Power]
  \begin{equation}
    \max{
      \sum_{s\in S} \rho_s \cdot \sum_{h\in H}[\pi_{s,h}\cdot g_{s,h}^W - \omega\cdot\pi_{s,h}\cdot|g_{s,h}^W - x_{h}^W|]
}
  \end{equation}
Bid constraints, output constraints
\end{tcolorbox}
The subscript $s,h$ means \textbf{in scenario $s$, in period $h$}
\begin{columns}
\column{0.5\linewidth}
\textbf{Constants}
\begin{itemize}
\item[$S, s$] Set and index of scenarios
\item[$H, h$] Set and index of hourly periods
\item[$\rho_s$] Probability of scenarios
\item[$\omega$] Penalty factor over the market price energy imbalances
\item[$\pi_{s,h}$] Expected market price in scenario $s$ period $h$
\end{itemize}

\column{0.5\linewidth}
\textbf{Variables}
\begin{itemize}
\item[$x_{h}^W$] Energy bid to the day-ahead market by the wind farm in period $h$
\item[$g_{s,h}^W$] Power output of the wind farm in scenario $s$ in period $h$
\end{itemize}
\end{columns}
\end{frame}

\begin{frame}
  \frametitle{Uncoordinated Operation (UO)}
\begin{tcolorbox}[colback=white!5,colframe=white!40!black,title= Model of Pumped-Storage Plant]
  \begin{equation}
    \begin{split}
    \max
      \sum_{s\in S} & \rho_s  \cdot  \sum_{h\in H}[\pi_{s,h}\cdot (g_{s,h}^p-d_{s,h}^p) -c^{su}\cdot y_{s,h} -c^{sd}\cdot z_{s,h} \\
     & -  \omega\cdot\pi_{s,h}\cdot|g_{s,h}^p -d_{s,h}^p- x_{h}^p|]
   \end{split}
  \end{equation}
\end{tcolorbox}
The subscript $s,h$ means \textbf{in scenario $s$, in period $h$}
\begin{columns}
\column{0.5\linewidth}
\textbf{Constants}
\begin{itemize}
\item[$c^{su},c^{sd}$] Start-up and shut-down costs of pumping units
\end{itemize}
\textbf{Variables}
\begin{itemize}
\item[$g_{s,h}^p$] Discharge power output of pumped-storage plant
\item[$x_{h}^p$] Energy bid to the DA market
\end{itemize}
\column{0.5\linewidth}
\begin{itemize}
\item[$d_{s,h}^p$] Pumping power input of the pumped-storage plant
\item[$y_{s,h},z_{s,h}$] (Integer)The number of units that are start-up and shut-down in the pumping mode
\end{itemize}
\end{columns}
\end{frame}

\subsection{Joint Optimization (JO)}
\begin{frame}
  \frametitle{Joint Optimization (JO)}
\begin{tcolorbox}[colback=white!5,colframe=white!40!black,title= Model of Pumped-Storage Plant]
  \begin{equation}
    \begin{split}
    \max
      \sum_{s\in S} & \rho_s  \cdot  \sum_{h\in H}[\pi_{s,h}\cdot (g_{s,h}^W + g_{s,h}^p-d_{s,h}^p) -c^{su}\cdot y_{s,h} -c^{sd}\cdot z_{s,h} \\
     & -  \omega\cdot\pi_{s,h}\cdot|g_{s,h}^W+g_{s,h}^p -d_{s,h}^p- x_{h}^{wp}|] \nonumber
   \end{split}
  \end{equation}
Bidding constraints, operation constraints
\end{tcolorbox}
The subscript $s,h$ means \textbf{in scenario $s$, in period $h$}
\begin{columns}
\column{0.5\linewidth}
\textbf{Variables}
\begin{itemize}
\item[$g_{s,h}^p$] Discharge power output of pumped-storage plant
\item[$g_{s,h}^W$] Power output of the wind farm
\item[$x_{h}^{wp}$] Joint bid to the DA market
\end{itemize}
\column{0.5\linewidth}
\begin{itemize}
\item[$d_{s,h}^p$] Pumping power input of the pumped-storage plant
\item[$y_{s,h},z_{s,h}$] (Integer)The number of units that are start-up and shut-down in the pumping mode
\end{itemize}
\end{columns}
\end{frame}


%----------------------------------------------------------------------------------------
\section{Numerial Results}
\subsection{Input Data}
%----------------------------------------------------------------------------------------
\begin{frame}
  \frametitle{Scenarios}

\begin{figure}[htbp]
  \centering
  \includegraphics[width = 0.7\linewidth]{scenarios.png}
  \caption{Scenarios}
  \label{fig:scenarios}
\end{figure}
Number of Price scenarios: 20

Number of Wind Production scenarios: 21

Total Scenarios: $21\times 20 = 420$
\end{frame}
%----------------------------------------------------------------------------------------
\subsection{Results}
\subsubsection{Operation Results}
\begin{frame}
  \frametitle{Obtained Optimal Hourly Bid}

\begin{figure}[htbp]
  \centering
  \includegraphics[width = \linewidth]{da_bids.png}
  \caption{Optained Hourly Bid}
\end{figure}
\end{frame}

\begin{frame}
\frametitle{Obtained Operation Through the Day in the Lower Reservoir For Each Scenario in the Joint Optimization}
\begin{figure}[htbp]
  \centering
  \includegraphics[width = 0.7\linewidth]{operation.png}
\end{figure}
\end{frame}
\subsubsection{Economic Resluts}
\begin{frame}
\frametitle{Economic Results}
\begin{figure}[htbp]
  \centering
  \begin{minipage}[htbp]{0.45\linewidth}
    \includegraphics[width = \linewidth]{profits.png}
    \caption{Profits (UO vs JO)}
  \end{minipage}
  \begin{minipage}[htbp]{0.45\linewidth}
    \includegraphics[width = \linewidth]{penalties.png}
    \caption{Penalties (UO vs JO)}
  \end{minipage}
\end{figure}
\begin{itemize}
\item Joint Operation has more profits.
\item Joint Operation has less penalties.
\end{itemize}
\end{frame}
\subsubsection{Interesting Results}
\begin{frame}
\frametitle{The influence of pumped-storage size}
\begin{figure}[htbp]
  \centering
  \begin{minipage}[htbp]{0.45\linewidth}
    \includegraphics[width=\linewidth]{profit_increase.png}
  \end{minipage}
  \begin{minipage}[htbp]{0.45\linewidth}
    \includegraphics[width=\linewidth]{penalty_reduction.png}
  \end{minipage}
\end{figure}
From CB1 to CB7: the capacity of pumped-storage plant increases.
\begin{itemize}
\item When the pumped-storage plant dimensions are small compared to the wind farm power, profits increase faster
\item When the value of the pumping capacity approaches that of the wind farm, profits remain almost constant.
\end{itemize}

\end{frame}

%----------------------------------------------------------------------------------------
\section{Conclusions and Reviews}
\subsection{Conclusions}
\begin{frame}
\frametitle{Conclusions and Future Work}
\begin{itemize}
\item In this paper, the author demonstrated that a joint short-term operation of a wind farm and of an isolated pumped-storage plant can be obtained by solving the presented optimization model.
\item Two-stage stochastic programming approach has proven to be an effective way to model the real decision making process that wind park operators face in a spot-market framework under uncertainty.
\item the models presented in this paper could be useful for assisting in investment decisions about new pumped-storage facilities.
\end{itemize}
Future research: introduce in the optimization model some risk-aversion measures.
\end{frame}
\subsection{Possible Problems}

\begin{frame}
\frametitle{Possible Problems}

\begin{itemize}
\item The scenarios (price and wind) are not convincing
  \begin{itemize}
  \item Price Scenarios are based on \textbf{``Modeling and forecasting electricity prices with input/output hidden Markov models,'' published on IEEE Trans. Power Syst., vol. 20, no. 1, pp. 1–12, Feb. 2005.}
  \item Not mention how to generate wind scenarios? The characterics of wind scenarios? Special scenarios?
  \end{itemize}
\item Some deep analysis?
\end{itemize}
\end{frame}

\end{document}